\documentclass[12pt,titlepage=true]{article}

\usepackage[utf8x]{inputenc}

\usepackage[frenchb]{babel}
\usepackage{amsmath}
\usepackage{amssymb}
\usepackage{amsfonts}
\usepackage{appendix}
\usepackage{array}
\usepackage{bigdelim}
\usepackage{bold-extra}
\usepackage{color}
\usepackage{colortbl}
\usepackage{caption}
\usepackage{easytable}
\usepackage{epsfig}
\usepackage{etex}
\usepackage{etoolbox}
\usepackage{fancyhdr} % en tête et pieds de page
\usepackage{graphicx}
\usepackage{here}
\usepackage{layout} % \layout au début de doc -> longueurs caractéristiques de la mise en page
\usepackage{makeidx}
\usepackage{mathrsfs}
\usepackage{mathtools}
\usepackage{multirow}
\usepackage{nopageno}
\usepackage{pslatex}
\usepackage{pstricks-add}
\usepackage[retainorgcmds]{IEEEtrantools} % des beaux tableaux (équations, matrice etc...) cf rapport TIPE
\usepackage[Gray, squaren]{SIunits} %éponyme, cf TIPE
\usepackage{shorttoc} %2e table des matières appelée par \shorttableofcontents{titre}{profondeur} cf rapport Stage FHM
\usepackage{textcomp} %polices
\usepackage[titles]{tocloft}
\usepackage{titlesec}
\usepackage{txfonts}
\usepackage{ucs}
\usepackage{wrapfig} % comme son nom l'indique cf rapTIPE
\usepackage{tocbasic}
\usepackage{lmodern}
%\usepackage[a4paper]{geometry}
%\usepackage{pdfpages}
%\usepackage[round]{natbib}
%\FrenchFootnotes
%\AddThinSpaceBeforeFootnotes
%\bibliographystyle{unsrt-fr}

\usepackage{xcolor}

%\usepackage{layout}
%\usepackage{rowcolor}
\definecolor{bleuX}{RGB}{0,62,92}
\renewcommand{\arraystretch}{1.3}
\renewcommand\labelitemi{\textbullet}
%modifchapter et numérotation
%\patchcmd{\chapter}{\thispagestyle{plain}}{}{}{} % laisse les entêtess
%\titleformat{\chapter}[hang]{\usefont{T1}{cmr}{bx}{sc}\LARGE }{\thechapter.}{0.5em}{}{} % On redéfinit le style, et les espacements
%\titlespacing{\chapter}{0pt}{-10pt}{20pt}
%\titleformat{\section}{\usefont{T1}{cmr}{bx}{n}\Large }{\thesection}{0.4em}{}{}
%\renewcommand\thechapter{\Roman{chapter}}
%\renewcommand\thesection{\thechapter.\arabic{section}}
\renewcommand\thesubsection{Question \arabic{section}.\arabic{subsection}}
%\renewcommand\thefigure{\arabic{figure}}
%\renewcommand\chapterpagestyle{fancy}
%modiftoc
%\renewcommand{\cftchappagefont}{\color{bleuX}\bfseries}
%\renewcommand{\cftsecpagefont}{\color{bleuX}}
%\renewcommand*\cftchapnumwidth{1.7em}
%\renewcommand*\cftsecnumwidth{2em}
%\renewcommand{\cftchapfont}{\usefont{T1}{cmr}{bx}{sc}\normalsize}
%modif entete et marge
%\geometry{hscale=0.75,vscale=0.76}
\setlength{\headheight}{15pt}
\addtolength{\textheight}{2.5cm}
\setlength{\hoffset}{-1.25cm}
\addtolength{\textwidth}{2cm}
\addtolength{\voffset}{0cm}

\patchcmd{\headrule}{\hrule}{\color{bleuX}\hrule}{}{}
\pagestyle{fancy}
	\fancyhead[L]{{\footnotesize\textsc{Lenormand} Augustin, \textsc{Bruneau} Basile}}
	\fancyhead[R]{}
	\fancyfoot[C]{{\color{bleuX}\bfseries\thepage}}
	
	
%modif appenix
%\makeatletter
%\patchcmd{\@chap@pppage}{\thispagestyle{plain}}{}{}{}
%\patchcmd{\@chap@pppage}{\Huge \bfseries \appendixpagename}{\usefont{T1}{cmr}{bx}{sc} \Huge \appendixpagename}{}{}
%\makeatother
%\renewcommand{\appendixpagename}{Annexes}
%\renewcommand{\appendixtocname}{Annexes}
%\renewcommand{\appendixname}{Annexe}

\newcommand{\esp}{\mathbb{E}}
\renewcommand{\exp}{\mathrm{e}^}


\title{Vitesse d'invasion pour un modèle de reproduction et dispersion}
\author{Augustin Lenormand \and Basile Bruneau}

\begin{document}
\maketitle

\section{Grandes déviations d'une marche aléatoire}
	\subsection{}
	\renewcommand\labelitemi{\textbullet}
	\begin{itemize}
	
	\item	Soit $\alpha< 1$ et $\lambda_1, \lambda_2$ dans $\mathbb{R}$.
	
			\begin{IEEEeqnarray*}{l C c}
				\esp( \exp{(\alpha \lambda_1 + (1 - \alpha)\lambda_2) X}) & = &  \esp((\exp{\lambda_1 X})^{\alpha}(\exp{\lambda_2 X})^{1-\alpha}\\
												   					  & \leqslant & (\esp(\exp{\lambda_1 X}))^{\alpha} (\esp(\exp{\lambda_2 X}))^{1- \alpha}
			\end{IEEEeqnarray*}
	
	
			Ici la dernière inégalité viens de l'inégalité de Hölder. En passant au logarithme il viens donc naturellement :	
	
			\begin{equation*}
				\Lambda((\alpha \lambda_1 + (1 - \alpha)\lambda_2) X) \leqslant \alpha \Lambda(\lambda_1 X) + (1-\alpha) \Lambda(\lambda_2 X)
			\end{equation*}
		
			\fbox{Donc $\Lambda$ est convexe.}
		
	\item	De même les fonctions $f_\lambda$ telles que $f_\lambda (x)=\lambda x - \Lambda (\lambda)$ sont toutes convexes. 
			Donc par définition leurs épigraphes sont convexes.
		
			Or l'épigraphe du supremum pour $\lambda$ dans $\mathbb{R}$ de ces fonctions est l'intersection des épigraphes de toutes ces fonctions. Comme intersection d'ensemble convexes , il est donc convexe lui aussi. Donc l'épigraphe de $\Psi$ est convexe.
		
			\fbox{Donc $\Psi$ est convexe.}

	\item	$\Lambda(0)=0$ donc :
			\begin{equation*}
			\forall x \in \mathbb{R}, \sup_{\lambda \in \mathbb{R}}(\lambda x - \Lambda(\lambda)) \geqslant 0 x - \Lambda(0) = 0
			\end{equation*}	
			\underline{Donc $\Psi\geqslant0$.}
			
			Soit $\lambda$ dans $\mathbb{R}$.
			\begin{equation*}
			\exp{f_\lambda(m))}=\frac{\exp{\lambda \esp(X)}}{\esp(\exp{\lambda X})}
			\end{equation*}			
			Or la fonction $x \mapsto \exp{\lambda x}$ est convexe, donc d'après l'inégalité de Jensen,	$\exp{\lambda \esp(X)}\leqslant\esp(\exp{\lambda X})$. Ainsi $\exp{f_\lambda(m))}\leqslant 1$ et $\lambda \esp(X) - \Lambda(\lambda)\leqslant 0$. 
			
			Donc $\Psi(\esp(X))\leqslant0$. Or $\Psi\geqslant0$.
			
			\fbox{Donc $\Psi$ admet un minimum en $m$ et $\Psi(m)=0$.}

	\item	Soit $x\geqslant m$ et $\lambda <0$. Alors on peut écrire :
			\begin{IEEEeqnarray*}{C}
			\lambda x -\Lambda(\lambda) \leqslant \lambda m - \Lambda(\lambda) = 0\\
			\lambda x -\Lambda(\lambda) \leqslant0 \leqslant\sup_{\lambda \in \mathbb{R}}(\lambda x - \Lambda(\lambda))
			\end{IEEEeqnarray*}
			
			\fbox{Donc prendre le supremum des $f_\lambda$ pour $\lambda\geqslant0$ est suffisant pour définir $\Psi$}
			

	\end{itemize}
		 
\end{document}